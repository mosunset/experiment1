\documentclass[a4j,12pt]{jsarticle}
\usepackage[dvipdfmx]{graphicx}
\usepackage{ascmac} %枠で囲むパッケージ
\usepackage{url}

\title{情報学群実験第1  最終レポート}
\author{1260277 浅野友哉}
\date{\today}

\begin{document}

\maketitle

\tableofcontents
\clearpage


% \part{部の名前}
% \section{節の名前}
% \subsection{小節の名前}
% \subsubsection{小小節の名前}
% \paragraph{段落の名前}
% \subparagraph{小段落の名前}


\section{概要}
このレポートでは、情報学群実験第1で開発したMinesweeper(地雷ゲーム)プログラムについて報告する。本レポートでは、Minesweeperの基本仕様と追加仕様について説明している。

追加仕様では、ゲームの盤面UIを改善し、昔のWindows版Minesweeperで使われていた文字色を再現し、背景色を変更する機能を追加している。さらに、ゲームの進行に応じて激励文を表示し、プレイヤーの楽しみを向上させている。また、新しいゲームを作成する際にタイルの数や地雷の数を指定できるように拡張し、柔軟なゲーム設定が可能となっている。

プログラムの開発には、再帰処理の実装が特に大変だったが、適切に実装することでスムーズなゲームプレイを実現している。また、拡張性を考慮しメソッドを随時分けた設計により、追加機能の実装が容易となっている。

報告書では、基本仕様・実現方法および追加仕様・実現方法について詳細に記述しており、得られた成果や課題に基づいた結論を示し、今後の展望を述べている。

\section{基本仕様・実現方法}

\subsection{仕様1: ゲーム開始時の盤面の初期化}
\subsubsection*{ゲーム開始時に,盤面にランダムに地雷を設置する}
雛形にあるように、盤面は二次元配列として実装されている。ゲーム開始時には、ランダムな場所に指定された数の地雷が配置される。そのあと同じ二次元配列内に、各タイルの周辺にいくつ地雷があるかを示す数字も計算される。タイル周辺の数字を計算する部分は、countAdjacentBombsメソッドとした。\\
※追加仕様4の実装により変更がある

\subsection{仕様2: タイルを開く処理}
\subsubsection*{パネルを左クリックした際,パネルを開く.}
タイルの表示状態・フラグの状態の二次元配列にその時の状態を保存しながら実行させていった。条件分岐により、盤面の範囲・タイルの表示状態・フラグの状態をチェックさせた。周辺に地雷がない場合、再帰処理により隣接するタイルも自動的に開かれる。openTileメソッド内で再帰処理させると、煩雑なのでopenAdjacentTilesメソッドに取り出した。openAdjacentTilesメソッドでも再帰処理による無限ループを防ぐために、openTileメソッドとほぼ同じ条件分岐により対応した。

\subsection{仕様3: 地雷を踏んだ際の処理}
\subsubsection*{パネルを開いた時,そのパネルに地雷が隠されていれば全てのパネルを開く.}
プレイヤーが地雷を踏んだ場合、ゲームオーバーとなる。gameEndのフラグをTrueにして、ResultDialogを出して、全ての地雷の位置を表示させる。gameEndのフラグが立っているので、どちらのクリックも聞かないようになる。

\subsection{仕様4: フラグの処理}
\subsubsection*{開かれていないパネルを右クリックした際,そのパネルに旗を立てる/取り除く.}
\subsubsection*{旗が立てられたパネルは,旗が取り除かれるまで左クリックで開けない.}
ゲームの開始前、最初の左クリックが行われるまで右クリックできないようにした。フラグ用の二次元配列を反転させていった。\\
仕様2と4で使用したsetTextToTileメソッドに引数を追加して、盤面表示である状態を変更させた。

\subsection{仕様5: 適切なダイアログの表示}
\subsubsection*{ゲームクリアもしくはゲームオーバーになった際,適切なダイアログを表示する.}
雛形にあるResultDialogをほぼそのまま使用した。追加仕様3で流用したため、ボタンの処理とサイズを変更した。

\section{追加仕様・実現方法}

\subsection{追加1: 盤面ののUIの変更}
\subsubsection*{文字についての変更}
昔のwindowsに搭載されていたMineSweeperの文字色を再現した。setTextToTileメソッドで条件分岐させて、色を変更させるのがむずかしかった。ゲーム終了後フラグが立っていた爆弾の場所は、黒塗りの星。そうでない場所は中抜きの星で表した。
\subsubsection*{背景色の変更}
初期の背景色を少し明るくし、ゲーム中に空いたタイルは、灰色に変更させた。ゲーム終了後すべてのタイルが表示されるとき、ゲーム中開いていないタイルは背景色を変更させないようにした。setTextToTileメソッド内での条件分岐でtilebgcolorメソッドを呼び出している。

\subsection{追加2: ゲームの状態表示等文字表示}
盤面の下に激励文を表示させるようにした。ゲームの盤面によって文章を変更させた。雛形に文章表示用のlabelを追加した。

\subsection{追加3: 新しいゲームの作成}
タイルの数・爆弾の数を指定して新しいゲームを開始できるようにした。タイルの総数-10個の爆弾の数を指定できる。その時とエラーは仕様5で使ったダイアログを流用した。\\
タイルどちらかの数でも少なくすると、盤面の表示や追加したUIがおかしくなるのがなおせなかった。ボタンのイベントリスナーに処理を追加して、Mainインスタンスを生成した。

\subsection{追加4: ゲーム開始時の爆弾配置}
ゲーム開始時最初の左クリック時に、そのマスが0になるようにした。コンストラクタの時点で初期化でなく、firstClickフラグクリックを判定して初期化している。initTableとsetBombsメソッド引数を追加して、クリック位置の爆弾設置判定をしている。


\section{まとめ・結論}

このレポートでは、情報学群実験第1で制作したMinesweeperのプログラムについて報告した。プログラムの作成には、特に再帰処理の実装や拡張性の考慮が重要であった。

プログラムの主な成果として、ゲームの盤面を開いた際に自動的に隣接するタイルも開く再帰処理を実装し、プレイヤーにスムーズな操作を提供することができました。また、UIの変更や激励文の表示など、追加仕様によってゲームのエンジョイメントを向上させました。

一方で、課題としては、タイルの数や爆弾の数を指定する際に発生するUIのおかしな挙動があった。この課題に対しては、入力値のバリデーションやUIの調整などを検討し、改善を図る必要がある。もう少し時間をかけて調整を行えば改善できると考えられる。

今後の展望としては、さらなる拡張や改善が考えられる。例えば、難易度選択やタイム記録の追加、グラフィカルな改善などが挙げられる。また、コードの可読性や効率性の向上も重要な課題となりうる。


\bibliographystyle{jplain}
\begin{thebibliography}{99}

\bibitem{javaawt} Oracle and/or its affiliates, "java.awt (Java Platform SE 8)", \url{https://docs.oracle.com/javase/jp/8/docs/api/java/awt/package-summary.html}, 7/23

\bibitem{javabook} 中山清喬,国本大悟:スッキリわかるJava入門 第3版インプレス 2019., 7/23

\bibitem{abs} JavaDrive, "Java | 絶対値を取得する(Math.abs)", \url{https://www.javadrive.jp/start/math/index1.html}, 7/23

\bibitem{javamen} "Container (Java Platform SE 6)", \url{https://www.ugs.kochi-tech.ac.jp/mendori/internal/docs/ja/api/java/awt/Container.html}, 7/23





\end{thebibliography}
\end{document}