\documentclass[a4j,11pt]{jarticle}

\title{情報学群実験第1 \LaTeX の復習}
\author{1260277 浅野友哉}
\date{\today}


\begin{document}

\maketitle

\section{本授業について}

情報学群実験第1では、Javaというプログラミング言語を用いて、プログラミングについて更に深く学習する。シラバスに書かれている通り、具体的な目標は以下の6つである。
\begin{enumerate}
    \item Javaアプリケーションを作成・実行できるようになる。
    \item 参照の概念をJavaプログラミングで実践できるようになる。
    \item クラスの継承・オーバーロード・例外をプログラミングに活用できるようになる。
    \item 代表的なソートアルゴリズムを理解し、プログラミングできるようになる。
    \item 代表的なデータ構造を理解し、プログラミングできるようになる。
    \item 小規模の仕様に基づき、完成したプログラムを構築できるようになる。
\end{enumerate}
\indent 本授業では、指定した教科書\cite{simplejava}に沿って授業を行う。教科書は比較的容易に記述されている。より深い学習のためには、技術的な詳細の載っている参考書を見ることも推奨する。

\section{文書作成について補足}
本文書を作成するにあたり、texファイルの1行目は
\begin{verbatim}
    \documentclass[a4j,11pt]{jarticle}
\end{verbatim}
としている。ページ幅等を合わせたい場合はそのようにすると良い。\\
\indent また、下記の参考文献リストは\verb|thebibliography|環境を用いている。今後レポート執筆の際に利用することになるので、使い方を学習しておくこと。


\bibliographystyle{junsrt}
\begin{thebibliography}{99}
\bibitem{simplejava} 中山清喬, 国本大悟著:スッキリわかる Java 入門.第3版, インプレス (2019).

\end{thebibliography}
\end{document}
